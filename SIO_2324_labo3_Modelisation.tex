% My standard header for TeX.SX answers:
\documentclass[12pt]{article}
\usepackage{amsmath}
\usepackage{geometry}
\usepackage[T1]{fontenc}
\usepackage{tgtermes}

\geometry{margin=2cm}

\title{SIO - Travail pratique 3}
\date{Janvier 2024}
\author{Hugo Huart \& Farouk Ferchichi}

\begin{document}
\maketitle

\section{Variables de décision et fonction objectif}

Afin de déterminer l'appartenance d'un objet à un groupe,
on utilise des variables binaires de décision $g_{ij}$ où $i=1,\ldots,n \text{ et } j=1,\ldots,m$
pour $n$ objets et $m$ groupes et où

\[
    g_{ij} =
    \begin{cases}
    1 \text{ si l'objet } i \text{ est dans le groupe } j  \\
    0 \text{ sinon}
    \end{cases}
\]
%
En connaissant les valeurs des objets données par $v_i$, où $v_i$ est la valeur de l'objet $i$, on
peut déterminer la valeur d'un groupe $j$ avec

\[
    \text{Valeur du groupe } j = \sum_{i=1}^{n}g_{ij} \cdot v_i
\]
%
À partir de cette information et du critère d'équité retenu ici consistant à maximiser
la plus petite valeur totale attribuée à un groupe, on peut établir la fonction objectif suivante

\[
    \max z = \min_{j} \sum_{i=1}^{n}g_{ij} \cdot v_i
\]
%
où $z$ est la plus grande valeur possible de la plus petite valeur attribuée à un groupe.
Cette fonction n'étant pas linéaire en l'état, on la linéarise à l'aide d'une
variable auxilliaire $t$. On remplace la somme minimale avec $t$ dans la fonction objectif, ce qui
donne

\[
    \max z = t
\]
%
Dans les contraintes, on va garantir que $t$ est plus petit ou égal à la plus petite valeur
attribuée à un groupe afin d'obtenir un résultat équivalent.

\pagebreak
\section{Contraintes}

On a un premier groupe de $m$ contraintes garantissant que $t$ est plus petit ou égal à la plus petite
valeur attribuée à un groupe

\[
    \tag{1} t \leq  \sum_{i = 1}^{n}{v_i \cdot g_{ij}} \text{,} \; \forall j = 1,\ldots,m
\]
%
On a également $n$ contraintes garantissant qu'un objet doit être attribué à exactement un seul
groupe. Ceci empêche d'avoir un objet attribué à plusieurs groupes ou d'avoir un
objet ne faisant partie d'aucun groupe

\[
    \tag{2} \sum_{j=1}^{m}{g_{ij} = 1} \text{,} \; \forall i = 1,\ldots,n
\]

\end{document}
